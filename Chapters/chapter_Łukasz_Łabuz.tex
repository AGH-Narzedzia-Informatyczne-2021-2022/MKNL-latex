\section{Łukasz Łabuz}

\graphicspath{{./Pictures}}

I added a picture of The Holy Cross Sermons (see \ref{fig:holycross})

\begin{center}
    \begin{figure} [h!]
        \centering
        \includegraphics[scale = 1.5]{Pictures/The_Holy_Cross_Sermons.png}
        \caption{These sermons are the oldest extant text in the Polish language}
        \label{fig:holycross}
    \end{figure}    
\end{center}

And here we can see a mathematical expression:

\begin{equation}
    \begin{cases}
        \frac {\partial u}{\partial t} - \triangle_x u = 0, & x \in \mathbb{R}^n, t \in \mathbb{R}_+ \\
        u(x,0) = g(x), & g:\mathbb{R}^n \to \mathbb{R}
    \end{cases}
\end{equation}

This system of differential equations describes the flow of heat. $g(x)$ is the initial distribution of heat, while $u(x,t)$ is the distribution of heat over time.

\bigskip

Table \ref{tab:table_ŁŁ} contains an example of a solved sudoku puzzle.

\begin{table}[H]
\centering
\begin{tabular}{lll|lll|lll}
\cellcolor{lightgray}7 & 9 & \cellcolor{lightgray}3 & 4 & \cellcolor{lightgray}8 & 5 & \cellcolor{lightgray}1 & 6 & \cellcolor{lightgray}2 \\
1 & \cellcolor{lightgray}2 & 6 & \cellcolor{lightgray}3 & 7 & \cellcolor{lightgray}9 & 8 & \cellcolor{lightgray}5 & 4 \\
\cellcolor{lightgray}8 & 4 & \cellcolor{lightgray}5 & 2 & \cellcolor{lightgray}1 & 6 & \cellcolor{lightgray}3 & 9 & \cellcolor{lightgray}7 \\ \hline
6 & \cellcolor{lightgray}5 & 4 & \cellcolor{lightgray}9 & 3 & \cellcolor{lightgray}7 & 2 & \cellcolor{lightgray}8 & 1 \\
\cellcolor{lightgray}2 & 7 & \cellcolor{lightgray}1 & 6 & \cellcolor{lightgray}4 & 8 & \cellcolor{lightgray}5 & 3 & \cellcolor{lightgray}9 \\
3 & \cellcolor{lightgray}8 & 9 & \cellcolor{lightgray}1 & 5 & \cellcolor{lightgray}2 & 7 & \cellcolor{lightgray}4 & 6 \\ \hline
\cellcolor{lightgray}9 & 3 & \cellcolor{lightgray}7 & 5 & \cellcolor{lightgray}2 & 4 & \cellcolor{lightgray}6 & 1 & \cellcolor{lightgray}8 \\
4 & \cellcolor{lightgray}1 & 8 & \cellcolor{lightgray}7 & 6 & \cellcolor{lightgray}3 & 9 & \cellcolor{lightgray}2 & 5 \\
\cellcolor{lightgray}5 & 6 & \cellcolor{lightgray}2 & 8 & \cellcolor{lightgray}9 & 1 & \cellcolor{lightgray}4 & 7 & \cellcolor{lightgray}3
\end{tabular}
\label{tab:table_ŁŁ}
\caption{Generated by \emph{sudokuweb.org}}
\end{table}

\bigskip

\textbf{Księga}\ldots Gdzieś w zaraniu dzieciństwa, o pierwszym świcie życia jaśniał horyzont od jej łagodnego światła. Leżała pełna chwały na biurku ojca, a ojciec, pogrążony w niej cicho,
pocierał \textit{poślinionym} palcem cierpliwie grzbiet tych odbijanek, aż ślepy papier zaczynał
mglić się, mętnieć, majaczyć błogim przeczuciem i z nagła złuszczał się kłakami bibuły
i odsłaniał rąbek \textcolor{teal}{pawiooki} i urzęsiony, a wzrok schodził, mdlejąc, w dziewiczy świt bożych
kolorów, w cudowną mokrość najczystszych \textcolor{red}{lazurów}.

O, to przetarcie się bielma, o, ta inwazja blasku, o błoga wiosno, o ojcze\ldots

Czasem ojciec wstawał od \colorbox{black}{\textcolor{white}{Księgi}} i odchodził. Wówczas zostawałem z nią \underline{sam na sam}
i wiatr szedł przez jej stronice i obrazy wstawały.

\begin{flushright}
Bruno Schulz, \emph{Sanatorium pod klepsydrą}
\end{flushright}

\bigskip

An example of list with many different bullets:

\begin{itemize}
    \item Element
    \item Inny
        \begin{enumerate}
            \item Numer
            \item Następny
            \item Kolejny
        \end{enumerate}
    \item Inne symbole
    \begin{itemize}
        \item[$\rightarrow$] Strzałka
        \item[$\uparrow$] W inną stronę
        \item[$\uparrow$] W tę samą 
    \end{itemize}
    \item Egzotyka
        \begin{itemize}
            \item[$\alpha)$] Alfa
            \item[$\beta)$] Beta
            \item[$\gamma)$] Gamma
        \end{itemize}
\end{itemize}