\section{Kacper Cienkosz}

\graphicspath{{./Pictures/}}

I added a picture of dog (see Figure \ref{fig:dog}).

\begin{center}
    \begin{figure}[hbpt]
        \centering
        \includegraphics[scale = 1.3]{dog.jpg}
        \caption{This dog is a good boy.}
        \label{fig:dog}
    \end{figure}
\end{center}

Table \ref{tab:tableKC} shows results of multiplication.

\begin{table}[H]
\centering
\begin{tabular}{|l||l|l|l|l|}
\hline
 multiply & 2 & 3 & 5 & 7 \\ 
 \hline\hline
 2 & 4 & 6 & 10 & 14 \\ 
 \hline
 3 & 6 & 9 & 15 & 21 \\ 
 \hline
 5 & 10 & 15 & 25 & 35 \\ 
 \hline
\end{tabular}
\label{tab:tableKC}
\caption{Caption of table.}
\end{table}

Equation \[ a + b = b + a \] \\

Theorem $ a^p \equiv a$ (mod p) \\

\par
Od kiedy \textbf{Przemysław Czarnek} ogłosił powstanie nowego przedmiotu, ciągle musi odpierać zarzuty opozycji. Tak było też podczas dzisiejszej audycji w Polskim Radiu. Czarnek pytany, czy nowy przedmiot będzie "hitem, czy kitem", odpowiedział, że "aby rozumieć rzeczywistość, trzeba wiedzieć, skąd się ona wzięła". Jego zdaniem wielu młodych ludzi \textit{"nie ma pojęcia, co się działo"} pod koniec II wojny światowej, pod koniec drugiej połowy XX w., zarówno w Polsce, Europie, jak i na całym świecie, a co za tym idzie, ludzie ci nie mogą rozumieć rzeczywistości, "która ich dzisiaj ogarnia". \par
Zapytany o to, czy HiT będzie \underline{"indoktrynował uczniów"} - jak sugeruje to opozycja - Czarnek odpowiedział: - Jeśli opozycja twierdzi, że to będzie indoktrynacja, to ja pytam opozycję na antenie, co rozumie pod pojęciem indoktrynacji, bo ja jako minister edukacji i nauki nie wiem jeszcze, jakie będą podstawy programowe i jaki będzie ostateczny program nauczania tego przedmiotu. Jeśli opozycja \emph{domaga} się indoktrynacji, to ja mam apel: proszę nie mierzyć innych swoją miarą. \\

\begin{flushright}
    \textbf{Źródło:} Onet
\end{flushright}

\begin{enumerate}
    \item First
    \item Second
    \item Third
\end{enumerate}

\begin{itemize}
    \item First
    \item Second
    \item Third 
\end{itemize}
