\section{Natalia Niedziałek}

\begin{center}
{\Large{\textbf{Trigonometric functions}}}
\end{center}

Here we can see some triangle. (see Figure~\ref{fig:triangle}).

\begin{figure}[H]
    \centering
    \includegraphics[width=0.8\textwidth]{Pictures/triangle.png}
    \caption{A right triangle}
    \label{fig:triangle}
\end{figure}

As in the table~\ref{tab:tablevalues} below:



\begin{table}[H]
\centering
\begin{tabular}{|l|lllll|}
    \hline
     & $0^{\circ}$ & $30^{\circ}$       & $45^{\circ}$       & $60^{\circ}$       & $90^{\circ}$ \\
     \hline
    cosx & 1           & $\frac{\sqrt3}{2}$ & $\frac{\sqrt2}{2}$ & $\frac{1}{2}$      & 0            \\
    sinx & 0           & $\frac{1}{2}$      & $\frac{\sqrt2}{2}$ & $\frac{\sqrt3}{2}$ & 1            \\
    tgx  & 0           & $\frac{1}{\sqrt3}$ & 1                  & $\sqrt3$           & -            \\
    ctgx & -           & $\sqrt3$           & 1                  & $\frac{1}{\sqrt3}$ & 0      \\ 
    \hline
\end{tabular}
\label{tab:tablevalues}
\caption{Table of values for trigonometric functions}
\end{table}




\par In mathematics, the \textbf{\underline{trigonometric functions}}  are real functions which relate an angle of a right-angled triangle to ratios of two side lengths. They are widely used in all sciences that are related to geometry, such as navigation, solid mechanics, celestial mechanics, geodesy, and many others. They are among the simplest periodic functions, and as such are also widely used for studying periodic phenomena through Fourier analysis.

\vspace{0,5cm}

\par The trigonometric functions most widely used in modern mathematics are the \textbf{\textit{sine}}, the \textbf{\textit{cosine}}, and the \textbf{\textit{tangent}}. Their reciprocals are respectively the cosecant, the secant, and the cotangent, which are less used. Each of these six trigonometric functions has a corresponding inverse function, and an analog among the hyperbolic functions.

\vspace{0,5cm}

Interesting trigonometric identity - Sum of sines with arguments in arithmetic progression:
if $\alpha \neq 0$

\vspace{0,5cm}

$sin\varphi + sin(\varphi+\alpha) + ... +sin(\varphi+n\alpha)=\frac{sin\frac{n+1}{2}\alpha\cdot sin(\varphi+\frac{n}{2}\alpha)}{sin\frac{1}{2}\alpha}$

\vspace{0,5cm}
Some basic exercises:
\begin{enumerate}
  \item The area of a right triangle is 50. One of its angles is 45°. Find the lengths of the sides and hypotenuse of the triangle.
  \item In a right triangle ABC, tan(A) = 3/4. Find sin(A) and cos(A).
  \item A rectangle has dimensions 10 cm by 5 cm. Determine the measures of the angles at the point where the diagonals intersect.
\end{enumerate}
\vspace{0,5cm}
Books worth reading:
\begin{itemize}
    \item[$\ast$] Trigonometry for Dummies
    \item[$\ast$]The Humongous Book of Calculus Problems
    \item[$\ast$]Must Know High School Trigonometry
\end{itemize}


