\section{Krzysztof Wójcik}

I added a photo of a \textit{Ambystoma mexicanum} (see Figure~\ref{fig:axolotl}).

\begin{figure}[H]
    \centering
    \includegraphics[width=0.3\textwidth]{Pictures/Axolotl.jpg}
    \caption{This is a really mischievous axolotl.}
    \label{fig:axolotl}
\end{figure}

Table \ref{tab:table_KW} shows some data about certain functions.

\begin{table}[H]
    \centering
    \begin{tabular}{|l|l|l|l|l|l|l|}
    \hline
    \(n\)   & 1 & 2 & 3 & 4  & 5   & 6   \\ \hline
    \(n^2\) & 1 & 4 & 9 & 16 & 25  & 36  \\ \hline
    \(2^n\) & 2 & 4 & 8 & 16 & 32  & 64  \\ \hline
    \(n!\)  & 1 & 2 & 6 & 24 & 120 & 720 \\ \hline
    \end{tabular}
    \label{tab:table_KW}
    \caption{Comparison of growth of different functions}
\end{table}

\par
In mathematics, the \textbf{Taylor series} of a function is an infinite sum of terms that are expressed in terms of the function's derivatives at a single point. For most common functions, the function and the sum of its Taylor series are equal near this point. Taylor's series are named after \textit{Brook Taylor}, who introduced them in \underline{1715}.
\par
If 0 is the point where the derivatives are considered, a Taylor series is also called a \textbf{Maclaurin series}, after \textit{Colin Maclaurin}, who made extensive use of this special case of Taylor series in the 18th century.\\
Source: \emph{Wikipedia.org}\\

Maclaurin series of the exponential function:

\[e^x = \sum_{n=0}^{\infty} \frac{x^n}{n!} = 1 + x + \frac{x^2}{2} + \frac{x^3}{6} + \frac{x^4}{24} + \cdots\]

\begin{enumerate}
    \item A
    \item B
    \begin{itemize}
        \item B1
        \item B2
        \item B3
    \end{itemize}
    \item C
    \begin{itemize}
        \item[$\rightarrow$] C1
        \item[$\rightarrow$] C2
        \item[$\rightarrow$] C3
    \end{itemize}
\end{enumerate}